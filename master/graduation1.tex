\documentclass[a4j, 12pt, dvipdfmx]{jsbook}

\usepackage[dvipdfmx]{graphicx}
%\usepackage[dvipdfmx]{hyperref}
\usepackage{cite}
\usepackage[dvipdfmx]{color}
\usepackage{url}
\usepackage{here}
\usepackage{ascmac}
\usepackage{fancybox}
\usepackage{listings}
\usepackage{multirow}
\usepackage{color}
\usepackage{algorithm}
\usepackage{algpseudocode}
\usepackage{amsmath}
\usepackage{subcaption}
\usepackage{longtable}
\usepackage{tcolorbox}
\usepackage{booktabs}
\usepackage{float}
\usepackage{subfig}
\definecolor{darkgray}{rgb}{.4,.4,.4}
\definecolor{purple}{rgb}{0.65, 0.12, 0.82}
	
\lstdefinelanguage{has}{
  language=Python,
  breaklines = true,
  basicstyle=\footnotesize\ttfamily,
  commentstyle={\textmc},
  keywordstyle=\color{black}\bfseries,
  showstringspaces=false,
  frame=tblr,
  numbers=left,
  stepnumber=1,
  numberstyle=\tiny,
  tabsize=2,
  stringstyle=\color{black}
}


%% \lstdefinelanguage{JavaScript}{
%%   keywords={typeof, new, true, false, catch, function, return, null, catch, switch, var, if, in, while, do, else, case, break},
%%   keywordstyle=\color{blue}\bfseries,
%%   ndkeywords={class, export, boolean, throw, implements, import, this},
%%   ndkeywordstyle=\color{darkgray}\bfseries,
%%   identifierstyle=\color{black},
%%   sensitive=false,
%%   comment=[l]{//},
%%   morecomment=[s]{/*}{*/},
%%   commentstyle=\color{purple}\ttfamily,
%%   stringstyle=\color{red}\ttfamily,
%%   morestring=[b]',
%%   morestring=[b]"
%% }

\lstdefinelanguage{JavaScript}{
  keywords={typeof, new, true, false, catch, function, return, null, catch, switch, var, if, in, while, do, else, case, break},
  keywordstyle=\bfseries,
  ndkeywords={class, export, boolean, throw, implements, import, this},
  ndkeywordstyle=\bfseries,
  sensitive=false,
  comment=[l]{//},
  morecomment=[s]{/*}{*/},
  commentstyle=\ttfamily,
  stringstyle=\ttfamily,
  morestring=[b]',
  morestring=[b]"
}

\lstset{
   language=JavaScript,
%%   backgroundcolor=\color{lightgray},
   extendedchars=true,
   basicstyle=\footnotesize\ttfamily,
   showstringspaces=false,
   showspaces=false,
   numbers=left,
   numberstyle=\footnotesize,
   numbersep=9pt,
   tabsize=2,
   breaklines=true,
   showtabs=false,
   captionpos=t,
   frame=single,
   texcl
}
	
\makeatletter
\def\Hline{
  \noalign{\ifnum0=`}\fi\hrule \@height 4.\arrayrulewidth \futurelet
  \reserved@a\@xhline}
\makeatother
\renewcommand{\lstlistingname}{リスト}
\renewcommand{\ttdefault}{pcr}

% \title{令和3年度 卒業論文\\題目「関数型プログラミング言語に対する\\型推論系に関する研究」}
% \author{情報科学部 メディア・ロボティクスコース\\ 2018311090 山田 桃香\\ 山本研究室}
% \date{}

\title{令和8年度 修士論文\\プログラミング演習における誤答分類のための\\構造的アプローチと意味的アプローチの比較評価}
\author{情報科学研究科 情報システム専攻\\ 2024831008 西村 優基\\ 山本研究室}
\date{}

\begin{document}
%\maketitle

\begin{titlepage}
\begin{centering}

\vspace*{3cm}
{\LARGE 令和8年度 修士論文}\\
\vspace*{1cm}
% {\LARGE 題目「関数型プログラミング言語に対する\\型推論系に関する研究」}\\
{\LARGE プログラミング演習における誤答分類のための\\構造的アプローチと意味的アプローチの比較評価}\\
\vspace*{6cm}

{\large 愛知県立大学大学院\\
情報科学研究科 情報システム専攻\\
2024831008 西村 優基\\
山本研究室}

\end{centering}
\end{titlepage}

\tableofcontents

\chapter{はじめに}

\section{背景}
プログラミング技術の習得は,現代社会においてますます重要性を増す.
その教育手法として,プログラミング演習やコンテストが活用されている.
これらの演習において,学習者が提出した大量の誤答コードを効率的に分析し,
適切なフィードバックを提供することは,学習効果の向上および指導者の負担軽減において重要な課題である.
しかし現状では,この誤答分析の多くは指導者の経験則に基づいた手作業で行われている.
卒業研究において,プログラミング演習における誤答コードの手動分類を試みた際にも,
その作業が非常に膨大な時間を要するだけでなく,分類基準が作業者の主観に依存,客観性を担保することが困難である課題があった.
したがって,プログラミング指導の効率化および質の均一化のためには,誤答分類作業を自動化する仕組みの構築が強く求められる.

\section{目的}
本研究の目的は,1.1で述べた誤答分類の自動化という課題を解決し,
それによりプログラミング学習の効果向上と指導者の負担軽減に寄与することにある.
本研究は,この課題の解決にあたり,学習データを必要とせず,かつコードの論理構造を決定論的に解析できる静的解析というアプローチに着目する.
上記のアプローチに基づき,本研究ではコードの静的解析と非階層クラスタリングを組み合わせた手法を提案し,その有効性を検証する.
さらに,この処理フローを解析から機械学習までを同一環境で行うプログラミング環境で構築することを目指す.
これにより,工学的な導入の容易さや,将来的に多様な機械学習手法へ展開する際の拡張性といった利点が得られると期待される.

\section{本論文の構成}
第2章で先行研究・関連研究を述べる.
%
第3章で提案する誤答分類の手法を述べる.
%
第4章で提案手法の効果を示すための実験方法を述べる.
%
第5章で実験結果について評価する.
%
第6章でまとめと今後の課題について述べる.




\chapter{関連研究}
\label{chap:related_work}

\section{プログラミングにおける誤答分析}
プログラミング教育において,学習者の解答コードを分析・分類する手法は,長年研究されてきた.これまでの研究におけるアプローチは,大きく分けて「構造的アプローチ」「意味的アプローチ」「動的・深層学習アプローチ」の3つに大別される.

まず,\textbf{構造的アプローチ}は,ソースコードのテキストや抽象構文木(以下AST)などの構文情報に着目する手法である.初期の研究ではテキストマイニング技術が用いられ,その後,木編集距離を用いた手法などが提案された.しかし,これらはインデントや変数名の違いといった表層的な変更に敏感であり,本質的なロジックの類似性を捉えにくいという課題があった.

次に,\textbf{意味的アプローチ}は,制御フローグラフ(以下CFG)やデータフローグラフ(以下DFG)を用いて,コードの動作やデータの流れといった「プログラムの意味」に着目する手法である.このアプローチはアルゴリズム的な戦略を捉えるのに適するが,グラフ間の同型性判定や編集距離計算など,計算コストが高くなる傾向にある.

最後に,\textbf{動的・深層学習アプローチ}は,実行時のトレース情報や機械学習モデルを活用する手法である.記述形式に依存しない柔軟な分類が可能であるが,大量の学習データの収集や,モデルごとの前処理が必要となる点が課題とされる.

次節以降では,本研究のアプローチと特に関連の深い,機械学習・LLMを用いた手法および,グラフ構造を用いた分類手法について詳述する.

\section{機械学習およびLLMを用いたコード分類}
近年では,機械学習や大規模言語モデル(以下, LLM)を用いてソースコードの特徴をベクトル空間へ埋め込み,分類を行う研究が進められている.

\subsection{機械学習による分類モデル}
深層学習を用いたソースコード解析の分野では, 藤原ら\cite{fujiwara2022}による研究が挙げられる. 彼らは, LSTMやGCNといったニューラルネットワークを用いてソースコードの特徴を学習し, 「ソースコード分類」と「コードクローン検出」の双方に取り組んでいる. ソースコード分類が単一のコード片を入力とし, その機能や属性のクラスを予測するタスクであるのに対し, コードクローン検出は2つのコード片を入力とし, その間の類似性や等価性を判定するタスクである. これらは入力形式や出力層の設計においては異なるアプローチをとるが, 深層学習モデルがコードの構文的・意味的特徴を捉える必要があるという点では本質的に同義である.

藤原らは, 従来のモデルが特定の学習データセットの特性に過剰に適合してしまう「データセット依存」の問題に着目した. 特にコードクローン検出において, 判定タスクを単純な2値分類から, ASTの類似度を予測する「回帰モデル」へと再定義することで, 未知のデータに対する汎化性能を向上させている.

しかしながら, 分類と検出のいずれのタスクであっても, こうした教師あり学習に基づく手法は, モデルの構築にあたり, 正解ラベルが付与された大量かつ高品質な学習データを必要とする点に変わりはない. プログラミング教育の現場において, 学習者が生成する多種多様な誤答コードに対し, 事前に網羅的なラベル付きデータを用意することはコスト的に困難である. また, モデルは学習した特徴空間内での判定には長けているが, 学習データに含まれない未知の論理的誤りや, 文脈に深く依存した意図の汲み取りにおいては限界が存在する.


\subsection{LLM活用の可能性}
\label{subsec:llm_potential}
近年のLLMは,ソースコードの生成や解説,要約といったタスクにおいて高い性能を示す.
この高度な文脈理解能力に基づけば,自然言語による指示(プロンプト)を与えることで,LLMがソースコード中の論理的な誤りや意図を汲み取り,適切なカテゴリへ分類できる可能性が期待される.
% 特に,従来の機械学習モデルでは学習データの作成コストが高かった「意味的なカテゴリ」への分類において,LLMの持つゼロショット・フューショット能力は強力な選択肢となり得る.

しかし,これらの能力が厳密な分類タスクにおいてどの程度有効に機能するかは自明ではない.
LLMには確率的な出力による不安定さや,事実に基づかない回答を生成するハルシネーションのリスクが含まれるためである.
したがって,プログラミングエラーの分類という正確性が求められるタスクにおいて,LLMが単独で実用的な精度を達成できるかについては,慎重な検証が必要である.

\section{グラフ構造を用いたコード分類}
\label{sec:asanas_cluster}
機械学習モデルとは対照的に,プログラムの構造的・意味的特徴を明示的なグラフとして捉え,分類を行う研究として,PaivaらによるAsanas Clusterがある\cite{Asanas Cluster}.
本研究の提案手法では,このアプローチを基礎として採用するため,本節ではその核心となるグラフ表現の定義および,先行研究において示された有効性について述べる.

\subsection{CFGとDFGによる特徴表現}
Asanas Clusterの最大の特徴は,コードの表層的なテキストではなく,アルゴリズム的戦略を捉えるために以下の2つのグラフ表現を利用する点にある.

\begin{itemize}
    \item \textbf{制御フローグラフ (CFG)}:\\
    Paivaらは, CFGを有向グラフ $G=(N,E,n_{0},n_{f})$ と定義する. 
    ここで, $N=\{n_{1},n_{2},...\}\cup\{n_{0},n_{f}\}$ はノードの集合であり, プログラムの「基本ブロック」に対応する. 
    $E$ は有向エッジの集合であり, ノード間の制御依存関係を表す. 
    また, $n_{0}$ と $n_{f}$ はそれぞれ, 制御がグラフに入力される「入口ノード」と, 制御がグラフから出る「出口ノード」を表し, これらは各ノードの後続ノードが最大2つになるように追加される.
    CFGは, プログラム実行中に取りうる経路や分岐といった制御フローの振る舞いを捉える構造であり, コードの記述スタイルの影響を受けにくいという利点がある\cite{Asanas Cluster}.

    \item \textbf{データフローグラフ (DFG)}:\\
    DFGは, 有向グラフ $G=(N,E)$ と定義される. 
    ここで, $N=\{n_{1},n_{2},...\}$ はノードの集合であり, 個々の計算単位や命令を表す. 
    $E$ は有向エッジの集合であり, データ依存関係を表す. 
    具体的には, エッジ $(n_{i},n_{j})$ ($n_{i},n_{j}\in N$) は, ノード $n_{i}$ の出力データがノード $n_{j}$ によって消費されることを意味する. 
    制御フローだけでは同一に見えるコードでも, 変数の使い回しや計算順序の違いなどデータの処理順序や依存関係が異なる場合, DFGによってその差異を明確に識別することが可能となる\cite{Asanas Cluster}.
\end{itemize}

\vspace{4cm}

\subsection{アプローチ概要と分類精度の評価}
Paivaらは,これらのグラフ構造の全てを比較するのではなく,グラフから構造的な\textbf{特徴量を抽出し,ベクトル空間上でクラスタリングを行う}手法を提案した.

具体的には,CFGからプログラムの複雑さや制御構造を,DFGから変数の操作頻度などを数値化し,それらをK-means法によって分類する.また,教育現場での利用を想定し,提出物が届くたびにモデルを更新する「インクリメンタル学習」を採用する点が特徴である.

\begin{figure}[t]
    \centering
    % 画像ファイル名は適切に変更してください
    \includegraphics[width=0.9\textwidth]{src/asanas_pca_result.png}
    \caption{先行研究におけるクラスタリング結果のPCA可視化(左: グラフ探索, 右: ソートアルゴリズム) (出典: Paiva et al. \cite{Asanas Cluster})}
    \label{fig:asanas_pca_result_prev}
\end{figure}


この手法の有効性は,以下の2つのデータセットを用いた実験によって検証されている.

\begin{enumerate}
    \item \textbf{グラフ探索アルゴリズム:} アルゴリズム設計の授業において収集された,深さ優先探索 (DFS) と幅優先探索 (BFS) の実装(計100件).
    \item \textbf{ソートアルゴリズム:} GitHubから収集された,異なる4種類のソートアルゴリズム(ヒープ,マージ,挿入,クイック)の実装(計100件).
\end{enumerate}

実験の結果,この手法は\textbf{異なるアルゴリズム戦略を完全に分離することに成功する}(Error Index = 0).
図\ref{fig:asanas_pca_result_prev}は,抽出された特徴ベクトルに対し主成分分析(PCA)を適用し,クラスタリング結果を2次元平面上に可視化したものである(出典:Paiva et al. \cite{Asanas Cluster}).

左図はグラフ探索(2クラスタ),右図はソートアルゴリズム(4クラスタ)の結果を示す.
図から明らかなように,\textbf{実装の詳細が異なるコードであっても,アルゴリズムの戦略が異なれば,特徴量空間上で明確に分離されたクラスタを形成する}ことが確認できる.


この結果は,CFGとDFGから抽出された特徴量が,コードの書き方ではなく「アルゴリズム」を捉えるのに十分な情報を有することを示唆する.

\section{本研究の位置づけ}
前節までに述べた関連研究の知見に基づき,本研究の立ち位置を明確にする.

PaivaらのAsanas Clusterは,図\ref{fig:asanas_pca_result_prev}に示した通り,正答コードにおける「アルゴリズム的戦略」の違いを明確にクラスタリングすることに成功する\cite{Asanas Cluster}.
本研究では,この\textbf{「コードの構造的な差異を捉える能力」が,誤答コードの分類にも応用可能である}という仮説に基づく.

誤答コードにおいても,その誤りの種類(ロジックの欠陥,ループ条件の誤り,変数の誤用など)によって,生成されるCFGやデータフローには特有の構造的特徴が現れる可能性が高い.
したがって,本研究ではAsanas Clusterで有効性が示された特徴抽出アプローチを「誤答コードの分類」というタスクに応用し,異なる誤りパターンを自動的に分離できるかを検証する.

その上で,第\ref{subsec:llm_potential}節で述べたLLMを用いた意味的な分類アプローチと比較を行うことで,誤答分析における最適な手法を明らかにする.


\chapter{分類手法の提案}
\label{chap:proposal}

% 本章では,プログラムソースコードからその構造的・意味的な特徴を抽出し,機能的な類似性に基づいて分類を行うための手法について詳述する.本研究では,複数のアプローチを検討するが,本節では先行研究であるAsanas Cluster \cite{Asanas Cluster}のアプローチに基づいた第一の手法について述べる.

本章では, ソースコードからその構造的・意味的な特徴を抽出し, 機能的な類似性に基づいて分類を行うための手法について詳述する. 本研究では, アプローチの異なる2つの手法(以下, 手法1および手法2)を採用し, 両者を比較検証する.

手法1は, 先行研究であるAsanas Cluster \cite{Asanas Cluster}のアプローチに基づき, ソースコードの制御フローやデータフローといった構造的特徴量を抽出し, クラスタリングアルゴリズムを用いる静的解析アプローチである. 対して手法2は, LLMの文脈理解能力を活用し, コードの意味内容に基づいて分類を行う生成AIアプローチである.

これら2つの手法を取り上げる理由は, 明確な数理的特徴に基づく従来の手法と, 近年急速に発展した意味論的解釈に基づく手法を対比させることで, 各アプローチの有効性と限界を明らかにするためである. 以下, 3.1節ではまず手法1の詳細について述べ, 続く節で手法2について説明する.

\section{手法1 : Asanas Clusterによる誤答分類}
\label{sec:method1}

本節では,提案手法の一つである「Asanas Cluster」について説明する.本手法は,ソースコードを静的解析し,その構造的特徴をベクトル化することで,教師なし学習によるクラスタリングを行うものである.
特筆すべき点として,本手法および先行研究は,コードの表層的な構文情報ではなく,アルゴリズムの「実行フロー」と「データの流れ」に着目する点が挙げられる.

\subsection{処理概要}
\label{subsec:processing_flow}

本提案手法の全体的な処理フローを図\ref{fig:processing_flow}に示す.
本システムは,入力されたソースコードに対して,以下4つの工程によりクラスタリング結果を出力する.

\begin{enumerate}
    \item 入力されたソースコードに対し,静的解析ツールを用いてプログラムの構造をグラフとして抽出する.
    \item 生成されたグラフ構造から,分類に有効な11種類の数値特徴量を抽出する.
    \item 抽出された特徴に対し,重要度に応じた重みづけと標準化を行い,数値ベクトルに変換する.
    \item ベクトル化されたデータに対し,インクリメンタルなK-meansアルゴリズムを適用し,類似したアルゴリズム戦略を持つ解答群を同一クラスタに分類する.
\end{enumerate}

\begin{figure}[htbp]
    \centering
    \includegraphics[width=0.9\textwidth]{src/asanas_flow.jpg}
    \caption{Asanas Clusterの処理概要図}
    \label{fig:processing_flow}
\end{figure}

\subsection{グラフ生成と解析アプローチ}
\label{subsec:graph_generation}

先行研究であるAsanas Cluster \cite{Asanas Cluster} では,解析基盤としてコードプロパティグラフ (CPG) と呼ばれる統合データ構造を採用する.
CPGはAST,CFG,DFGを包含する詳細なグラフであり,先行研究ではここから評価順序グラフ (EOG) を経てCFGを生成し,さらにデータフロー情報を組み合わせることで特徴量を算出する.

一方,本研究では,分類に必要な特徴量が最終的に「CFGによる構造情報」と「データフローによる依存関係」の2点に集約されることに着目した.
そのため,CPGという包括的なグラフ構造を明示的に生成する中間工程は実装の複雑化を招くため省略し,ソースコードから直接的にCFGの構築およびデータフロー解析を行うアプローチを採用する.これにより,先行研究と同等の情報をより軽量な処理で抽出することを目的とする.

具体的な解析対象は以下の2点である.

% \subsubsection{CFGの構築}
\paragraph{CFGの構築} \mbox{}\\
プログラムの実行順序を表現するグラフである.
先行研究では,詳細なノード粒度を持つCPGからCFGを生成するため,分岐や合流を持たない連続した命令を収縮させてブロック化する後処理を必要としていた.

これに対し本手法では,基本ブロックを単位とする標準的なCFG構築手法を採用する.基本ブロックとは,内部に分岐や合流を含まない一連の命令列を指すため,グラフ生成を行った段階で既に冗長な遷移は集約されている.したがって,先行研究で行われている収縮処理は,本手法の生成プロセスにおいては自動的に達成されるため,明示的な後処理としては不要となる.これにより,アルゴリズムの骨格(分岐,ループなど)を直接的かつ効率的に抽出する.なお,本手法によって生成されるグラフの妥当性および基本ブロックによる集約の確認については,第\ref{sec:graph_validation}節の実験設定にて実例を用いて詳述する.

% \subsubsection{データフロー解析}
\paragraph{データフロー解析} \mbox{}\\
変数間の定義と使用のつながりを追跡する解析である.
CFGだけでは捉えきれない「値の流れ」を解析することで,制御構造が類似していてもデータの処理手順が異なるアルゴリズムの識別を可能にする.

\subsection{特徴量の定義とベクトル化}
\label{subsec:feature_definition}

グラフ構造(CFGおよびDFG)同士の類似度を直接算出するグラフ編集距離等の手法は,計算コストが極めて高い.そこで本手法では,先行研究 \cite{Asanas Cluster}のアプローチに倣い,グラフ構造から抽出したスカラー値による特徴ベクトルを用いる手法を採用する.

具体的には,先行研究で有効性が示されている表\ref{tab:features}の定義に従い,CFGからプログラムの複雑さを,データフロー解析の結果から変数の利用特性を抽出し,合計11次元の特徴ベクトルを構成する.

\begin{table}[h]
    \centering
    \caption{Asanas Clusterにおける特徴量一覧と重みづけ \cite{Asanas Cluster}}
    \label{tab:features}
    \begin{tabular}{llcc}
        \toprule
        \textbf{特徴量} & \textbf{詳細} & \textbf{グラフ} & \textbf{重み} \\
        \midrule
        connected\_components & 制御フローグラフ内の連結成分数 & CFG & 1.0 \\
        loop\_statements & ループ文(for, while等)の数 & CFG & 1.0 \\
        conditional\_statements & 条件分岐文(if等)の数 & CFG & 1.0 \\
        cycles & グラフ内のサイクルの数 & CFG & 1.0 \\
        paths & グラフ内の異なるパスの数 & CFG & 1.0 \\
        cyclomatic\_complexity & サイクロマティック複雑度 & CFG & 1.0 \\
        \midrule
        variable\_count & 使用されている変数の総数 & DFG & 0.6 \\
        total\_reads & 変数読み取り操作の総数 & DFG & 0.1 \\
        total\_writes & 変数書き込み操作の総数 & DFG & 0.1 \\
        max\_reads & 単一変数に対する最大読み取り数 & DFG & 0.1 \\
        max\_writes & 単一変数に対する最大書き込み数 & DFG & 0.1 \\
        \bottomrule
    \end{tabular}
\end{table}

\paragraph{特徴量の選定意図と重みづけ} \mbox{}\\
各特徴量に対する重みづけについても,先行研究における設定を踏襲する.
先行研究では,アルゴリズムの戦略においては命令の実行順序が最も支配的であるという知見に基づき,CFG由来の特徴量に対してそれぞれ $1.0$ の重みを付与する.一方で,データフロー情報は補助的な特徴と位置づけられ,合計でCFG特徴量1つ分となるように重みが分散されている.これは,変数の個数や操作回数がアルゴリズムの本質よりも実装スタイルに依存しやすい傾向を考慮した設計である.

\paragraph{特徴量の前処理と選定} \mbox{}\\
\label{subsec:preprocessing}
抽出された11次元の特徴ベクトルに対し,クラスタリング精度を高めるための前処理を行う.このプロセスについても,基本的に先行研究の手順に準拠する.

\paragraph{特徴量の選定} \mbox{}\\
特徴量間の冗長性を排除するため,データセットを用いて各特徴量ペアのピアソンの相関係数を算出する.先行研究で採用されている基準と同様に,相関係数が $0.9$ を超えるペアが存在する場合,投票形式により冗長な特徴量を削除する判定を行う(なお,本実験データセットにおいては先行研究のデータを参考にするため該当ペアは存在しなかった).

\paragraph{インクリメンタルな正規化} \mbox{}\\
各特徴量のスケールを統一するため,平均0,分散1への正規化(標準化)を行う.本システムは提出物が逐次入力されるオンライン処理を想定するため,全データの統計量を事前に固定するのではなく,各特徴量について実行平均と実行分散を更新しながら正規化を行う実装とする.

\subsection{クラスタリングアルゴリズム}
\label{subsec:clustering_algorithm}

前節で定義した\textbf{11次元の特徴量ベクトル}を入力とし,K-means法を用いたクラスタリングを行う.
本研究では,Paivaらによる先行研究 \cite{Asanas Cluster}において提案されたインクリメンタルK-meansアルゴリズムを全面的に採用する.
なお,本手法におけるクラスタリング計算は11次元の特徴量空間上で直接行われるものであり,可視化のために用いられる2次元平面上の操作ではない.

\subsubsection{採用したアルゴリズムと処理フロー}

本手法で用いるクラスタリング処理の擬似コードを Algorithm \ref{alg:clustering} に示す.これは先行研究 \cite{Asanas Cluster}で定義された処理フローを再現したものである.

\begin{algorithm}[t]
\caption{インクリメンタルK-meansアルゴリズム (文献\cite{Asanas Cluster}より採用)}
\label{alg:clustering}
\begin{algorithmic}[1]
\Require $k$: クラスタ数 ($2 \le k \le 16$)
\Require $C$: ランダムに初期化された$k$個のセントロイド集合(各セントロイドは11次元ベクトル)
\Require $N$: 各クラスタの要素数配列(0で初期化)
\Procedure{ProcessSubmission}{$S$}
    \State \Comment{$S$ は新規提出コードから抽出された11次元特徴量ベクトル}
    \State $min\_dist \gets \infty$
    \State $min\_c \gets \text{null}$

    \State \Comment{1. 11次元空間におけるユークリッド距離を用いて最近傍セントロイドを探索}
    \For{each centroid $c \in C$}
        \State $d \gets \text{EuclideanDistance}(c, S)$
        \If{$d < min\_dist$}
            \State $min\_dist \gets d$
            \State $min\_c \gets c$
        \EndIf
    \EndFor

    \State \Comment{2. クラスタ統計情報を更新}
    \State $N[min\_c] \gets N[min\_c] + 1$

    \State \Comment{3. セントロイド位置を更新 (オンライン更新)}
    \If{$S$ is Correct}
        \State $\eta \gets 1 / N[min\_c]$ \Comment{学習率の算出}
        \State $min\_c \gets min\_c + \eta \times (S - min\_c)$ \Comment{ベクトル演算によりセントロイドを $S$ の方向へ移動}
    \EndIf

    % \State \Return $min\_c$
\EndProcedure
\end{algorithmic}
\end{algorithm}

% \vspace{15mm}

このアルゴリズムは,新規の提出データ $S$ がシステムに入力されるたびに実行され,以下の3つのステップで処理が進む.

\begin{enumerate}
    \item 事前に$k$個のセントロイド $C$ をランダムに初期化しておく.新規入力 $S$ に対し,\textbf{11次元の特徴量空間上での}ユークリッド距離を用いて最も近いセントロイド $min\_c$ を特定する.
    \item 特定されたクラスタに属する要素数 $N$ をインクリメントする.
    \item 入力 $S$ が正解コードである場合のみ,モデルの学習を行う.具体的には,そのクラスタの要素数に応じた学習率 $\eta = 1/N$ を算出し,セントロイドmin\_cを $S$ の方向へ $\eta$ の割合だけ移動させる.
\end{enumerate}

\subsubsection{アルゴリズムおよびパラメータの選定理由}
本研究において上記のアルゴリズムおよび設定を採用した理由は,先行研究における以下の検証結果に基づく.

\paragraph{インクリメンタル手法の採用理由} \mbox{}\\
通常のK-means法を本システムのように学生が順次解答を提出する環境に適用した場合,新たなデータが追加されるたびに全データを用いたモデルの再学習が必要となる.これでは計算コストが肥大化し,学生への即時フィードバックに必要な応答速度を維持できないため適さない. これに対し先行研究では,データ到着ごとに重心を更新するオンライン学習アプローチによって,大規模な再計算を行うことなく,低コストかつリアルタイムにモデルを最新状態に維持できることが示されているため,これを採用した \cite{Asanas Cluster}.

\paragraph{ユークリッド距離の選定理由} \mbox{}\\
最近傍探索における距離尺度としてユークリッド距離を選定した理由は,アルゴリズム戦略の分離精度にある.先行研究では,クラスタリングの精度評価として誤分類の度合いを示すエラー指標が用いられている.その結果,マンハッタン距離では0.3,コサイン距離では0.25であったのに対し,ユークリッド距離では0となり,誤分類が発生しないことが報告されている.本研究においても高い分類精度を確保するため,この結果に基づき\textbf{多次元特徴量空間における}ユークリッド距離を採用する \cite{Asanas Cluster}.

\paragraph{クラスタ数 $k$ の設定} \mbox{}\\
クラスタ数 $k$ の上限を16とするのは,一般的なプログラミング課題において想定される正解アルゴリズム戦略の種類が16を超えることは稀であるという先行研究の知見に基づいている \cite{Asanas Cluster}.


\subsection{誤答分類への適用と対応点}
Asanas Cluster \cite{Asanas Cluster}は, 正解コードの構造的特徴を抽出してアルゴリズム戦略ごとに分類することを目的としていた. これに対し本研究では, 同一の手法を「誤答コードの分類」に適用する.
本手法が誤答のパターン分類にも有効であると考える理由は, 抽出される11次元の特徴量がプログラムの「構造」と「データの流れ」を定量化しているためである.

具体的に, 誤答コードにおけるバグや実装ミスは特徴量ベクトル上の差異として現れると仮定できる.
例えば, ループ条件の誤りや分岐の欠落といった論理的なバグはCFGの形状を変化させるため, 特徴量である \texttt{loop\_statements}や \texttt{cyclomatic\_complexity}の値が, 正解コードや他の種類の誤答とは異なる挙動を示すと考えられる.
また, 変数の初期化忘れや誤った代入といったデータ操作に関するバグはDFGにおける依存関係を変化させるため, \texttt{total\_reads}や\texttt{max\_writes}といったデータフロー特徴量の変動として捉えることが可能である.

したがって, 本研究では先行研究における「アルゴリズム戦略の差異によるクラスタ形成」という概念を, 「誤答原因(バグの種類)の差異によるクラスタ形成」に対応させて適用する. これにより, 構文的な違いに左右されず, 構造的・意味的な誤りの傾向に基づいて誤答を分類可能であるかを検証する.

\section{手法2 : LLMによる誤答分類}
本節では,もう一つの提案手法のとなるLLMを用いた誤答分類システムについて\\詳述する.

\subsection{分類のアプローチ}
本手法では,ソースコードの意味理解において優れた性能を示すLLMに対し,「問題文」「誤答コード」「分類カテゴリ」の3要素を入力として与え,そのコードが定義されたカテゴリのいずれに該当するかを判定させる.
静的解析や単純なパターンマッチングなど従来の手法とは異なり,LLMはコードの文脈や意図を深く理解できるため,論理的な誤りや複雑なバグに対しても高精度な分類が期待できる.

\subsection{モデル選定}
本研究では,誤答分類を行う基盤モデルとして,OpenAI社が提供するGPT-4oを採用した.
2025年現在,より推論能力に特化した OpenAI o1 や,コーディング生成能力に優れた Claude 3.5 Sonnet などのモデルが存在するが,本研究の目的である「教育的フィードバックのための誤答分類」においては,以下の3点の技術的・実用的観点から GPT-4o が最適であると判断した.

\begin{enumerate}
    \item \textbf{指示従順性と出力形式の安定性:}
    本手法のプロンプトでは,後段のシステムで回答を機械的に処理するため,「間違いの説明」の後に「カテゴリ番号」のみを出力するという特定のフォーマットを遵守する必要がある.
    GPT-4o は複雑な指示に対する従順性が極めて高く,従来のモデルで頻発した「余計な会話文の混入」や「出力順序の逸脱」といったフォーマット違反が少ない\cite{openai_structured_2024}.これにより,正規表現等を用いた分類ラベルの抽出処理を安定して行うことが可能となり,システム化に適する.

    \item \textbf{分類タスクにおける適合率の高さ:}
    教育現場におけるフィードバックでは,正解を誤って不正解と判定する「誤検知」が学習者の混乱を招くため,極力避ける必要がある.既存のベンチマーク評価によると,GPT-4o は分類タスクにおいて Claude 3.5 Sonnet と比較して高い適合率(Precision 86.2\%)を示す傾向が報告されている \cite{vellum_2025}.生成能力(コードを書く力)においては他モデルが優位な場合もあるが,誤りの識別能力(コードを読む力)においては GPT-4o が本タスクに対しより保守的かつ適切な特性を持つ.

    \item \textbf{推論速度とコストのスケーラビリティ:}
    推論特化型モデル(OpenAI o1 等)は高い論理性能を持つ反面,応答に数十秒の時間を要するため,将来的なリアルタイム・フィードバック・システムへの応用には不向きである.
    GPT-4o は旧世代モデルと比較して約2倍の生成速度を持ちながら,十分な論理推論能力(MATHベンチマーク 76.6\%)を維持する \cite{sentisight_2025}.また,同一アーキテクチャの軽量モデル(GPT-4o mini)への蒸留が可能であり,将来的な大規模展開におけるコストパフォーマンスの向上が見込める.
\end{enumerate}

\subsection{プロンプト設計}
LLMの性能を最大限に引き出し,かつ揺らぎの少ない出力を得るため,プロンプトの設計には細心の注意を払った.
実際に使用するプロンプトの構成をリスト\ref{list:prompt_design}に示す.

% \begin{figure}[t]
%     \centering
%     % widthで行幅いっぱい,colback=whiteで背景白,colframe=blackで枠黒
%     % boxrule=0.5ptで細い枠線,sharp cornersで角を直角に
%     \begin{tcolorbox}[width=0.95\textwidth, colback=white, colframe=black, boxrule=0.5pt, sharp corners, fontupper=\ttfamily\small]
% このコードの間違いを教えてください.\\
% 最後に以下のカテゴリのどれに当たるか数字のみ出力してください.\\
% 間違いの説明とカテゴリのみ出力してください.\\

% 【問題文(必要最低限の情報)】\\
% (ここに問題文を記述)\\

% 【カテゴリ】\\
% 1. [カテゴリ1]\\
% 2. [カテゴリ2]\\
% ...\\
% n. [カテゴリn]\\

% 【コード】\\
% (ここに誤答コードを記述)\\
%     \end{tcolorbox}
%     \caption{提案手法2で使用するプロンプト構成}
%     \label{fig:prompt_design}
% \end{figure}

\begin{lstlisting}[
    caption={提案手法2で使用するプロンプト構成},
    label={list:prompt_design},
    language={},
    escapechar=|,        % |で囲った部分を通常のLaTeXとして処理する設定
    basicstyle=\ttfamily\small,
    frame=single         % 枠線が必要な場合
]
|このコードの間違いを教えてください.|
|最後に以下のカテゴリのどれに当たるか数字のみ出力してください.|
|間違いの説明とカテゴリのみ出力してください.|

|【問題文(必要最低限の情報)】|
|(ここに問題文を記述)|

|【カテゴリ】|
|1. [カテゴリ1]|
|2. [カテゴリ2]|
|...|
|n. [カテゴリn]|

|【コード】|
|(ここに誤答コードを記述)|
\end{lstlisting}

プロンプト設計においては,以下の工夫を取り入れた.

\begin{itemize}
    \item \textbf{入力情報の最適化}:問題文に含まれるストーリー要素などのノイズを除去し, 制約条件や入出力形式のみを抽出した. これにより, LLMがアルゴリズムの本質的理解や論理構造の解析に集中できるよう入力を最適化する.

    \item \textbf{Chain-of-Thoughtによる推論精度の向上}:カテゴリ番号の出力に先立ち「誤りの説明」を生成させることで, Chain-of-Thought(思考の連鎖)効果を導入した. まず誤りを言語化させるプロセスを経ることで, 推論の根拠を明確にし, 分類精度の向上を図っている.

    \item \textbf{出力形式の厳格化}:最終的な出力を所定のフォーマットに限定することで,後段のシステムにおける機械的なパースを容易にし,安定したシステム動作を担保する.
\end{itemize}



\chapter{誤答分類実験}
\label{chap:methodology}

\section{実験目的}
本研究の目的は,プログラミング演習における誤答ソースコードを機械的に分類し,教育支援に役立てることである.
本実験では,そのための具体的な手法として,従来手法である静的解析に基づくクラスタリング手法Asanas Clusterと,近年急速に発展するLLMを用いた分類手法の2つを取り上げ,それぞれの分類精度および特性を比較検証する.具体的に,同一の誤答データセットに対して両手法を適用し,どちらの手法がより高精度に誤答パターンを分類できるかを定量的に評価する.この比較を通じて,誤答コードの自動分類システムを構築する上で,どちらのアプローチがより実用的かつ有効であるかを明らかにすることを目的とする.

\section{実験環境}
本実験は,以下のハードウェアおよびソフトウェア環境下で実施した.
実験の再現性を担保するため,詳細なバージョン情報を記載する.

\subsection{ハードウェア・OS構成}
\begin{itemize}
    \item \textbf{OS}: Windows 11
    \item \textbf{実行環境}: WSL2 (Windows Subsystem for Linux 2) - Ubuntu Environment
\end{itemize}

\subsection{ソフトウェア・ライブラリ}
静的解析に基づく特徴抽出,クラスタリング,および評価指標の算出には,プログラミング言語Pythonを用いた.
主要な使用ライブラリとその用途は以下の通りである.

\begin{itemize}
    \item \textbf{Python}: 3.12.3
    \item \textbf{PyJoern}: ソースコードの静的解析エンジン,CFGおよびDFGの生成に使用.
    \item \textbf{NetworkX}: グラフ構造の解析.循環的複雑度やパス数の算出に使用.
    \item \textbf{scikit-learn}: K-meansクラスタリングの実装,および評価指標(適合率,再現率,F値)の計算に使用.
    \item \textbf{SciPy}: ハンガリアンアルゴリズムの実装.クラスタリング結果と正解ラベルの最適マッチングに使用.
    \item \textbf{NumPy / Pandas}: 数値計算およびデータ処理に使用.
\end{itemize}

また,比較対象となる大規模言語モデルを用いた分類には,OpenAI社が提供するAPIを使用した.

\begin{itemize}
    \item \textbf{使用モデル}: GPT-4o
\end{itemize}

\section{対象データ}
本実験では,オンラインジャッジシステム「AtCoder」上で公開されている問題セット「競プロ典型90問」を対象とした.
この問題セットは,競技プログラミングにおいて頻出するアルゴリズムや実装テクニックを網羅しており,誤答パターンの分析に適すると考えられる.

\subsection{選定問題とデータ規模}
「競プロ典型90問」の中から, 難易度★2に設定されている問題計10問を選定した.
各問題について, 判定結果が「WA (Wrong Answer)」または「TLE (Time Limit Exceeded)」となった提出コードを収集した. データの収集にあたっては, 1つの問題につき100件の誤答コードを抽出した. この際, コーディングスタイルの多様性を確保するため, 同一ユーザーによる提出は重複して取得せず, 100名の異なるユーザーによるソースコードを対象とした.

その結果, 本実験では10種類の問題それぞれについて100件ずつ, 合計10セットの独立したデータ群を使用する.本研究における分析および評価は, これら10セットのデータ群に対して個別に実施するものであり, 異なる問題のデータを混合して扱うことはない.

また, 各データ群は「評価用正解データ」兼「実験用入力データ」として利用する.具体的に, 後述する手順で全データに対して手動で分類ラベルを付与し, 実験時にはこのラベルを隠蔽した状態で各手法に入力する.最終的に, 各手法が出力した分類結果を正解ラベルと照合することで, 分類精度の定量的評価を行う.

\subsection{データのフィルタリングと選定基準}
収集した誤答コードの中には,本研究が目的とする「アルゴリズムや論理の誤り」の分析に適さないデータが含まれている可能性があるため,以下の基準に基づきフィルタリングを行った.

\begin{enumerate}

    \item \textbf{デバッグ出力によるWAの除外}: AtCoderのジャッジシステムでは, 解答と無関係な標準出力が含まれる場合, たとえ計算ロジックが正しくても「WA」と判定される. 本研究の目的はアルゴリズムや論理の誤りを分析することであるため, このような「ロジックは正しいがデバッグ出力の消し忘れによってWAとなったコード」はノイズと見なし, 収集段階で除外した.
  
    \item \textbf{出力形式ミスの除外}:
    アルゴリズムや解法自体は正しいものの,空白区切りで出力すべきリストを,Pythonのリスト形式のまま出力するなど出力形式の不備によってWAとなっているコードは,本研究の分析対象である論理的な誤答ではないため排除した.

    \item \textbf{コンパイル可能性}:
    構文エラーで実行不可能なコードは対象外とし,コンパイルおよび実行が可能でありながら,論理的な誤りまたは実行時間超過が発生するコードのみを対象とした.

    \item \textbf{コメントの削除}:
    前処理として,コード内のコメント記述をすべて削除した.
    手法1においてはコメントの有無は解析結果に影響しないが,手法2においては,コメント内にアルゴリズムのヒントが含まれていた場合に推論が有利になったり,逆にコメントアウトされたデバッグコードを論理的な誤りとして誤認したりする可能性がある.これらのバイアスを排除し,純粋なコードロジックのみによる公平な比較を行うために適用した.
\end{enumerate}

\subsection{正解ラベルの作成}
収集した誤答コードに対する評価の基準を作成するため,各問題のソースコードを目視で確認し,手動による分類を行った.
具体的な手順として,まず収集したコードに見られる具体的な誤りを洗い出し,それらを誤答の原因や種類に基づいて一般化することで,各問題に対して3〜5個程度の主要な誤答カテゴリを定義した.
各問題において定義した誤答カテゴリの内訳を表\ref{tab:error_categories}に示す.

\begin{longtable}{|l|p{10cm}|}
    \caption{各問題における誤答カテゴリの定義}
    \label{tab:error_categories} \\

    % --- 最初のページのヘッダー ---
    \hline
    \textbf{問題ID} & \textbf{定義した誤答カテゴリ} \\
    \hline
    \endfirsthead

    % --- 2ページ目以降のヘッダー(続きとわかるようにする) ---
    \multicolumn{2}{l}{\footnotesize (前ページの続き)} \\
    \hline
    \textbf{問題ID} & \textbf{定義した誤答カテゴリ} \\
    \hline
    \endhead

    % --- 各ページの終わりのフッター ---
    \hline
    \multicolumn{2}{r}{\footnotesize 次ページへ続く} \\
    \endfoot

    % --- 最終ページの終わりのフッター ---
    \hline
    \endlastfoot

    % --- データ本体 ---
    \textbf{aa} &
    1. ハッシュ衝突による誤り \newline
    2. 二分探索の実装およびロジックの間違い \newline
    3. 出力形式または出力内容の間違い \newline
    4. 名簿管理における操作ミス \newline
    5. 重複チェック処理における条件漏れ \\
    \hline

    \textbf{ag} &
    1. 点灯個数の探索・カウント処理の誤り \newline
    2. 特定条件下の考慮漏れ \\
    \hline

    \textbf{bc} &
    1. 全探索によるTLE \newline
    2. 組み合わせ選択後の計算ロジックの誤り \newline
    3. 組み合わせの作成・選択自体の間違い \newline
    4. 計算以外の制約条件確認ミス \newline
    5. 剰余計算処理の誤り \\
    \hline

    \textbf{bi} &
    1. 出力時のインデックス操作ミス \newline
    2. 山札の操作・管理ミス \\
    \hline

    \textbf{bo} &
    1. $N=0$ 等のコーナーケース処理漏れ \newline
    2. 再帰関数による計算エラー \newline
    3. 出力形式または内容の間違い \newline
    4. $n$進数変換処理の実装ミス \\
    \hline

    \textbf{bz} &
    1. グラフ探索の実装不備によるTLE \newline
    2. 「頂点がちょうど1個」の条件判定ミス \newline
    3. 隣接リスト・配列作成時のミス \\
    \hline

    \textbf{d} &
    1. 出力時のインデックス操作ミス \newline
    2. 累積和の作成ロジックの誤り \newline
    3. 累積和を使用していない \newline
    4. 累積和を使用するが計算方法に誤りがある \\
    \hline

    \textbf{j} &
    1. インデックス以外のクエリ処理ミス \newline
    2. クエリ処理時のインデックス操作ミス \newline
    3. 累積和の作成ロジックの誤り \newline
    4. 累積和を使用していない \newline
    5. 最終的な結果の計算ミス \\
    \hline

    \textbf{v} &
    1. 最大公約数の算出ミス \newline
    2. 浮動小数点演算による誤差 \newline
    3. GCDを使用しない場合の最小回数算出ミス \newline
    4. GCDを使用した場合の最小回数算出ミス \\
    \hline

    \textbf{x} &
    1. 判定条件の記述漏れ \newline
    2. 条件判定ロジックの誤り \newline
    3. 差分計算におけるミス \\

\end{longtable}


\section{グラフ生成の妥当性と基本ブロックの確認}
\label{sec:graph_validation}

第\ref{chap:proposal}章で述べた通り,本研究の手法1ではソースコードから直接CFGを生成し,基本ブロック単位での解析を行っている.
この生成プロセスの妥当性,および先行研究で行われていた収縮処理が本手法の生成プロセスにおいて自動的に達成されていることを確認するため,本実験環境における解析対象のソースコード例をリスト\ref{list:cfg_code}に,それに対応して生成されたCFGを図\ref{fig:cfg_example}に示す.

\begin{lstlisting}[caption={入力ソースコード例}, label={list:cfg_code}, basicstyle=\ttfamily\small, frame=single, language=Python]
def classify_sum(a, b):
    x = a * 10
    y = b + 5
    total = x + y

    if total > 50:
        print("Large")
    else:
        print("Small")
\end{lstlisting}

\begin{figure}[tbp]
    \centering
    % グラフの横幅は見やすさに応じて調整してください
    \includegraphics[width=0.85\textwidth]{src/pyjoern_cfg.png}
    \caption{PyJoernによって生成されたCFG(リスト\ref{list:cfg_code}に対応)}
    \label{fig:cfg_example}
\end{figure}

\vspace{3cm}

リスト\ref{list:cfg_code}は単純な条件分岐を含むソースコードである.
これに対応する図\ref{fig:cfg_example}を確認すると,変数定義である \texttt{x=10}, \texttt{y=20}, \texttt{total=x+y} という一連の連続した命令が,個別のノードに分割されることなく,単一の基本ブロックとして統合されていることがわかる.
また,そのブロックから \texttt{if} 文による条件分岐が正しくエッジとして表現されている.

この結果から,本実験の環境において,先行研究が手動で行っていた「冗長な遷移の収縮」が基本ブロックの構築として内包されていることが確認できる.したがって,生成されたグラフはアルゴリズムの構造的特徴を抽出するために十分な抽象度と妥当性を有すると言える.


\section{実験手順}
本実験では,前節で述べた誤答コード群に対し,静的解析に基づく手法とLLMに基づく手法のそれぞれを適用し,分類性能を検証する.
両手法において,最終的な出力と正解ラベルの照合プロセスが異なるため,以下にそれぞれの手順を詳述する.

\subsection{手法1: Asanas Cluster}
本手法は,ソースコードから抽出した特徴量に基づきK-means法により分類を行い,その結果を正解カテゴリに割り当てるアプローチである.

\subsubsection{1. 特徴量の抽出および前処理}
分類の入力データとしては,前章の表\ref{tab:features}で定義したCFGおよびDFGに基づく計11次元の特徴量ベクトルを用いる.

特にpathsの算出においては,ループや分岐構造を適切に反映させるため,「各ノードを最大2回まで訪問してよい」という制約ルールを設けている.

なお,K-means法は特徴量の値の範囲(スケール)の違いによる影響を受けやすいため,先行研究と同様に,前処理としてすべての特徴量が平均0,分散1となるように標準化を行った上で実験に使用した.


\subsubsection{2. K-meansによるクラスタリング}
標準化されたデータに対し,K-means法を適用して分類を行った.クラスタ数 $K$ は,各問題に対して事前に定義した正解カテゴリ数(2〜5個)と一致するように設定した.
この段階で,各コードは機械的に \texttt{cluster1}, \texttt{cluster2} \dots といったクラスタIDに分類される.


\subsubsection{3. ハンガリアンアルゴリズムによる正解ラベルとの対応付け}
本実験では,K-meansの初期値として各正解カテゴリのセントロイドを与えてクラスタリングを行う.
しかし,クラスタリングの過程で重心位置が更新されるため,最終的に出力されるクラスタIDと正解ラベルの対応関係は必ずしも初期状態と一致するとは限らない.

そこで,分類精度を正しく評価するために,形成された各クラスタと正解カテゴリの最適な対応付けを行う必要がある.本実験では,単純な「貪欲法」ではなく,割り当て問題の最適解を導く「ハンガリアンアルゴリズム」を採用した.
以下に,競合がない理想的なケースと,競合が発生するケースの双方において,両アルゴリズムがどのような挙動を示すかを比較する.

\paragraph{ケース1:競合がない場合} \mbox{}\\
各クラスタの特徴が明確で,特定のカテゴリにデータが集中している理想的な状態を表\ref{tab:greedy_success}に示す.

\begin{table}[t]
  \centering
  \caption{競合がない場合のデータ例}
  \label{tab:greedy_success}
  \begin{tabular}{lcc}
    \toprule
      & \textbf{カテゴリA} & \textbf{カテゴリB} \\
    \midrule
    \textbf{クラスタ0} & \textbf{100} & 10 \\
    \textbf{クラスタ1} & 5 & \textbf{90} \\
    \bottomrule
  \end{tabular}
\end{table}

\vspace{3cm}

\begin{itemize}
    \item 貪欲法では,クラスタ0はA(100)を選択し, クラスタ1はB(90)を選択する. 競合しないため, スムーズに最適解(190件)に到達する.
    \item ハンガリアンアルゴリズムでは,全組み合わせを探索するが, 当然「クラスタ0$\to$A, クラスタ1$\to$B」の組み合わせがコスト最大と判定される.
\end{itemize}
このような単純なケースでは, 両手法の結果は完全に一致する.
つまり, ハンガリアンアルゴリズムを用いても貪欲法の利点は損なわれない.

\paragraph{ケース2:競合がある場合} \mbox{}\\
次に, 誤答傾向が似通っており, 複数のクラスタで特定のカテゴリが競合している場合を表\ref{tab:greedy_fail}に示す.


\begin{table}[b]
  \centering
  \caption{競合がある場合のデータ例}
  \label{tab:greedy_fail}
  \begin{tabular}{lcc}
    \toprule
      & \textbf{カテゴリA} & \textbf{カテゴリB} \\
    \midrule
    \textbf{クラスタ0} & \textbf{100} & 99 \\
    \textbf{クラスタ1} & 90 & 5 \\
    \bottomrule
  \end{tabular}
\end{table}

\begin{itemize}
    \item 貪欲法では,まずクラスタ0に着目し, 僅差であっても多い方の\textbf{カテゴリA}(100)を確保してしまう. その結果, クラスタ1には本来割り当てるべきA(90)を割り当てられず, 残った\textbf{カテゴリB}(5)が強制的に割り振られる. \\
    $\to$ 総正解数: $100 + 5 = 105$
    
    \item ハンガリアンアルゴリズムでは,「クラスタ0でAを諦めてBを取る($100 \to 99$)」という局所的な損失を受け入れることで, 「クラスタ1でAを取る($5 \to 90$)」という大域的な利益を優先する判断を行う. \\
    $\to$ 総正解数: $99 + 90 = 189$
\end{itemize}

競合が発生した場合, 貪欲法は著しく精度を落とすが, ハンガリアンアルゴリズムは全体最適を保つことができる.


\paragraph{まとめ} \mbox{}\\
以上の比較から, ハンガリアンアルゴリズムは「競合がない場合は貪欲法と同等の結果」を保証しつつ, 「競合がある場合は貪欲法よりも優れた結果」を出すことができる手法であると言える.
本研究の誤答データは性質が近似し競合が発生しやすいため, 安全かつ高精度な評価を行うためにハンガリアンアルゴリズムを採用した.




\subsection{手法2: LLMによる分類手順}
本手法は,LLMにコードと分類基準を提示し,どのカテゴリに該当するかを直接判断させるアプローチで,以下3つの手順により分類を行う.

\begin{enumerate}
    \item \textbf{プロンプトの構築}:
    各問題のソースコードと,前節で定義した「誤答カテゴリのリスト」をプロンプトとして構築した.
    LLMに対し,入力されたコードがリスト内のどのカテゴリに最も適合するかを選択させた.

    \item \textbf{分類結果の出力}:
    LLMは,コードの内容を解析し,その誤答原因に合致すると判断したカテゴリIDを出力する.ここでは,物理的なディレクトリ移動は行わず,各ファイルに対する予測ラベルとして記録した.

    \item \textbf{正解データとの照合}:
    LLMによる予測結果の正当性を検証するため,ファイル名をキーとした正解ラベルの辞書データを作成し,照合を行った.
    具体的には,各ファイルが実際に所属すべき正解カテゴリと,LLMが出力したカテゴリIDが一致するかを確認した.
\end{enumerate}

\section{評価指標}
本実験では,手法1と手法2の分類性能を定量的に比較するため,以下の指標を用いる.
分類結果の妥当性を評価するため,手動で付与した正解カテゴリを「正解」,各手法によって分類された結果を「予測」として定義する.

\subsection{適合率 (Precision)}
適合率は, あるカテゴリとして「予測」されたデータのうち, 実際にそのカテゴリが「正解」であったデータの割合である.
予測結果の中に,どれだけノイズが含まれていないかを示す.

\begin{equation}
  \text{Precision} = \frac{\text{正しく予測されたファイル数}}{\text{当該カテゴリと予測された全ファイル数}}
\end{equation}

この値が1に近いほど,予測結果に含まれるノイズが少ないことを意味する.

\subsection{再現率 (Recall)}
再現率は, あるカテゴリが「正解」であるデータのうち, どれだけ正しく「予測」できたかを表す割合である.
本来検出スべきデータをどれだけ網羅できたかを示す.

\begin{equation}
  \text{Recall} = \frac{\text{正しく予測されたファイル数}}{\text{当該カテゴリが正解である全ファイル数}}
\end{equation}

この値が1に近いほど,取りこぼしが少ないことを意味する.

\subsection{F値 (F1-score)}
適合率と再現率の調和平均である. 精度の高さ(ノイズの少なさ)と網羅性(取りこぼしのなさ)のバランスを表す総合的な指標となる.

\begin{equation}
  \text{F1} = 2 \times \frac{\text{Precision} \times \text{Recall}}{\text{Precision} + \text{Recall}}
\end{equation}

適合率と再現率のどちらか一方が極端に低い場合,F値は大きく低下するため,バランスの取れた分類ができているかを判定するのに適する.


\chapter{結果と考察}
\section{実験結果}
前節の実験設定に基づき,手法1および手法2を用いて分類を行い,各評価指標を算出した結果を表\ref{tab:simple_grid},\ref{tab:average_scores}に示す.

{ % スコープを限定するための括弧開始
\small % 文字サイズを小さくする (または \footnotesize)
\renewcommand{\arraystretch}{1.1} % 行間を少し詰める (通常は1.0)
\begin{longtable}{|c|l|c|c|}
\caption{各問題カテゴリにおける評価指標} \label{tab:simple_grid} \\

% --- 最初のページのヘッダー ---
\hline
\textbf{問題} & \textbf{カテゴリ} & \textbf{手法1} & \textbf{手法2} \\
\hline
\endfirsthead

% --- 2ページ目以降のヘッダー(続きとわかるようにする) ---
% \multicolumn{2}{l}{\footnotesize (前ページの続き)} \\
\hline
\textbf{問題} & \textbf{カテゴリ} & \textbf{手法1} & \textbf{手法2} \\
\hline
\endhead

% --- 各ページの終わりのフッター ---
\hline
\multicolumn{2}{r}{\footnotesize (次ページへ続く)} \\
\endfoot

% --- 最終ページの終わりのフッター ---
\hline
\endlastfoot



% --- AA (5行) --- やっぱ同じ構造のコードは分類ムズイ
\multirow{5}{*}{\textbf{AA}}
 & カテゴリ1 & 0.40 & 0.91 \\ \cline{2-4}
 & カテゴリ2 & 0.50 & 0.57 \\ \cline{2-4}
 & カテゴリ3 & 0.36 & 0.77 \\ \cline{2-4}
 & カテゴリ4 & 0.00 & 0.35 \\ \cline{2-4}
 & カテゴリ5 & 0.53 & 0.62 \\ \hline

% --- AG (2行) ---
\multirow{2}{*}{\textbf{AG}}
 & カテゴリ1 & 0.77 & 0.57 \\ \cline{2-4}
 & カテゴリ2 & 0.97 & 0.98 \\ \hline

% --- BC (5行) ---
\multirow{5}{*}{\textbf{BC}}
 & カテゴリ1 & 0.00& 0.33 \\ \cline{2-4}
 & カテゴリ2 & 0.46 & 0.26 \\ \cline{2-4}
 & カテゴリ3 & 0.17 & 0.27 \\ \cline{2-4}
 & カテゴリ4 & 0.44 & 0.67 \\ \cline{2-4}
 & カテゴリ5 & 0.77 & 0.67 \\ \hline

% --- BI (2行) ---
\multirow{2}{*}{\textbf{BI}}
 & カテゴリ1 & 0.82 & 0.65 \\ \cline{2-4}
 & カテゴリ2 & 0.00 & 0.83 \\ \hline

% --- BO (4行) ---
\multirow{4}{*}{\textbf{BO}}
 & カテゴリ1 & 0.67 & 0.82 \\ \cline{2-4}
 & カテゴリ2 & 0.10 & 0.36 \\ \cline{2-4}
 & カテゴリ3 & 0.00 & 0.11 \\ \cline{2-4}
 & カテゴリ4 & 0.13 & 0.00 \\ \hline

% --- BZ (3行) ---
\multirow{3}{*}{\textbf{BZ}}
 & カテゴリ1 & 0.50 & 0.68 \\ \cline{2-4}
 & カテゴリ2 & 0.69 & 0.00 \\ \cline{2-4}
 & カテゴリ3 & 0.53 & 0.40 \\ \hline

% --- D (4行) ---
\multirow{4}{*}{\textbf{D}}
 & カテゴリ1 & 0.00 & 0.33 \\ \cline{2-4}
 & カテゴリ2 & 0.29 & 0.22 \\ \cline{2-4}
 & カテゴリ3 & 0.48 & 0.44 \\ \cline{2-4}
 & カテゴリ4 & 0.40 & 0.50 \\ \hline

% --- J (5行) ---
\multirow{5}{*}{\textbf{J}}
 & カテゴリ1 & 0.22 & 0.67 \\ \cline{2-4}
 & カテゴリ2 & 0.66 & 0.87 \\ \cline{2-4}
 & カテゴリ3 & 0.38 & 0.83 \\ \cline{2-4}
 & カテゴリ4 & 0.31 & 0.94 \\ \cline{2-4}
 & カテゴリ5 & 0.00 & 0.67 \\ \hline

% --- V (4行) ---
\multirow{4}{*}{\textbf{V}}
 & カテゴリ1 & 0.33 & 0.53 \\ \cline{2-4}
 & カテゴリ2 & 0.39 & 0.58 \\ \cline{2-4}
 & カテゴリ3 & 0.17 & 0.66 \\ \cline{2-4}
 & カテゴリ4 & 0.63 & 0.83 \\ \hline

% --- X (3行) ---
\multirow{3}{*}{\textbf{X}}
 & カテゴリ1 & 0.63 & 0.83 \\ \cline{2-4}
 & カテゴリ2 & 0.00 & 0.71 \\ \cline{2-4}
 & カテゴリ3 & 0.42 & 0.77 \\ \hline

\end{longtable}
}

\begin{table}[b]
  \centering
  \caption{問題ごとの平均スコアおよび全体平均}
  \label{tab:average_scores}
  \begin{tabular}{|l|c|c|}
    \toprule
    \textbf{問題ID} & \textbf{手法1} & \textbf{手法2} \\
    \midrule
    \textbf{AA} & 0.36 & \textbf{0.64} \\
    \textbf{AG} & \textbf{0.87} & 0.78 \\
    \textbf{BC} & 0.37 & \textbf{0.44} \\
    \textbf{BI} & 0.41 & \textbf{0.74} \\
    \textbf{BO} & 0.23 & \textbf{0.32} \\
    \textbf{BZ} & \textbf{0.57} & 0.36 \\
    \textbf{D}  & 0.29 & \textbf{0.37} \\
    \textbf{J}  & 0.31 & \textbf{0.80} \\
    \textbf{V}  & 0.38 & \textbf{0.65} \\
    \textbf{X}  & 0.35 & \textbf{0.77} \\
    \midrule
    \textbf{全体平均} & 0.38 & \textbf{0.57} \\
    \bottomrule
  \end{tabular}
\end{table}

表\ref{tab:simple_grid},\ref{tab:average_scores}の結果を見ると,多くの問題において手法1と比較して,手法2の方が高いスコアを示す傾向が見られた.
特に問題JやXにおいては,手法2が全体を通して安定して高い分類精度を記録する.その一方で,問題AGなど一部のケースにおいては,手法1のスコアが手法2を上回る,あるいは同等の高い値を示す.

\section{考察}
実験結果のスコアをもとに手法1と手法2のそれぞれスコアが高く,精度の高い分類ができた要因とスコアが低く精度の低い分類ができた要因またはLLMの実験結果で注意深い結果を確認できたので考察していく.
なお,本研究では誤答コードの性質を解析対象とするため,提出者のプライバシー保護および不利益の防止を目的として,すべてのデータは匿名化処理を行い,個人が特定されない形で使用する.

\subsection{手法1による分類が有効であった事例と考察}
\label{subsec:case_study_ag}

手法1において,特に高い精度で誤答コードの分類が行えた事例として,問題AG(033 - Not Too Bright \cite{typical90_ag})を取り上げる.

\subsubsection{問題概要と分類結果}
問題AGは,$H \times W$ のグリッド上にLEDを配置する際,「任意の $2 \times 2$ の領域内に2つ以上のLEDが点灯していてはならない」という制約下で,点灯可能なLEDの最大個数を求める問題である.

本問題に対する手法1の適用結果を図\ref{fig:asanas_pca_result}に示す.
図の左側は手動で分類した正解ラベルに基づく分布, 右側は手法1によって自動分類された結果である.
この図は,抽出された11次元の特徴量ベクトルを主成分分析(PCA)を用いて2次元に圧縮・可視化したものであり,点同士の空間的な距離が近いほど,コードの構造的特徴が類似していることを意味する.

\begin{figure}[t]
    \centering
    \includegraphics[width=1.0\textwidth]{src/ag_tea.png}
    \caption{問題AGにおける特徴量分布の可視化(左:正解ラベルに基づく分布, 右:手法1による分類結果)}
    \label{fig:asanas_pca_result}
\end{figure}


\newpage

左図(正解分布)を確認すると,データは空間上で明確に2つの集団に分離していることがわかる.
これに対応して,右図(手法1)においても,その分布形状をなぞるように2つのクラスタが形成されており,手法1がコードの構造的な差異を正しく検知し,分離に成功していることが視覚的に確認できる.

解析の結果,この空間的な分離は,表\ref{tab:error_categories}で定義した以下の2つの誤答カテゴリの違いを反映している.

\begin{enumerate}
    \item \textbf{点灯個数の探索・カウント処理の誤り}: 動的計画法(DP)や全探索を用いて配置をシミュレートしようとし,実行時間制限超過(TLE)となるパターン.
    \item \textbf{特定条件下の考慮漏れ}: 数学的考察により計算式を導出したが,特定の条件($H=1$ または $W=1$)を考慮できていないパターン.
\end{enumerate}


% \newpage

\subsubsection{具体的な誤答コードの比較}
表\ref{tab:error_categories}で定義されたカテゴリ間で,コード構造にどのような差異があるかを具体的に確認する.

ソースコード\ref{list:ag_tle}は,カテゴリ「点灯個数の探索・カウント処理の誤り」に分類された典型的なコードである.
このコードでは,2次元配列 dp を用意し,2重ループを用いてグリッド全体を探索する.
この手法は,H,Wの問題の制約に対して計算量が過大となり,TLEを引き起こす.
構造的特徴として,ネストされたループ構造や,配列への頻繁なアクセスが挙げられる.

\begin{lstlisting}[caption={カテゴリ1:点灯個数の探索・カウント処理の誤り}, label={list:ag_tle}, language=Python]
def notTooBright(H: int, W: int) -> int:
    dp = [[False] * (W + 1) for _ in range(H + 1)]
    total = 0
    for r in range(1, H + 1):
        for c in range(1, W + 1):
            if not dp[r - 1][c - 1] and not dp[r - 1][c] and not dp[r][c - 1]:
                dp[r][c] = True
                total += 1
    return total

def main():
    H, W = map(int, input().split())
    print(notTooBright(H, W))

if __name__ == "__main__":
    main()
\end{lstlisting}

一方,ソースコード\ref{list:ag_wa}は,カテゴリ「特定条件下の考慮漏れ」に分類されたコードである.
ここではループ処理を用いず,数式によって直接解を算出する.
この計算式自体は正しいが,行または列が1の場合($H=1$ または $W=1$)にはすべてのマスに配置可能であるというコーナーケースの考慮が漏れており,WAとなる.
このカテゴリのコードは,カテゴリ1とは対照的に,制御構造が極めて単純であり,演算処理が支配的であるという特徴を持つ.

\begin{lstlisting}[caption={カテゴリ2:特定条件下の考慮漏れ}, label={list:ag_wa}, language=Python]
h, w = map(int, input().split())

print(((h+1)//2) * ((w+1)//2))
\end{lstlisting}

上記2つのコード例から明らかなように,この問題における2つの誤答カテゴリは,コードの構造に決定的な差異がある.
表\ref{tab:error_categories}の定義におけるカテゴリ1は「多重ループと条件分岐」を主体とするのに対し,カテゴリ2は「単純な算術演算」のみで構成されている.
手法1で用いている静的解析特徴量は,このような構造的な違いを鋭敏に捉えることができる.
その結果,特徴量空間において両者が大きく離れた位置に分布し,K-means法によるクラスタリングが高精度に機能したと考えられる.
したがって,誤答の種類によってアルゴリズムの選択や実装方針が根本的に異なる場合,本手法であるAsanas Clusterアプローチは極めて有効である.

ただし,本手法は「コードの構造的な類似性」に基づいて分類を行っているため,カテゴリ2のように「数式は合っているが,特定の条件分岐が抜けている」といった,行レベルでの具体的なバグ箇所の特定までは行えていない点には留意が必要である.



\subsection{手法1による分類が困難であった事例と考察}
\label{subsec:case_study_010}
前節で述べた問題AGのような成功例がある一方で,多くの問題では明確な分離が困難であった.
その代表例として,「競プロ典型90問 - 010 Score Sum Queries \cite{typical90_j}」(以下,問題J)を取り上げ,静的解析ベースの手法の限界について考察する.

\subsubsection{問題概要と分類結果}
問題Jは,$N$ 人の生徒がクラス1またはクラス2に所属しており,与えられた $Q$ 個のクエリ(範囲 $[L_i, R_i]$)に対して,各クラスの点数合計を回答する問題である.
制約上,単純なループによる集計では間に合わず,累積和を用いた $O(1)$ または $O(\log N)$ でのクエリ処理が必須となる.

本問題に対する手法1の適用結果を図\ref{fig:010_pca_result}に示す.
図の左側は正解ラベルに基づく分布, 右側は手法1による分類結果である.
目視による定性評価では,本問題の誤答コードは主に表\ref{tab:error_categories}に示す5つのカテゴリに分類された.

\begin{enumerate}
    \item \textbf{インデックス以外のクエリ処理ミス}: 入力受け取りまでは行っているが,クエリごとの計算ロジックが破綻する,あるいは記述されていない.
    \item \textbf{クエリ処理時のインデックス操作ミス}: 累積和は作成できているが,区間計算時に $L$ と $R$ を逆にする等のミスがある.
    \item \textbf{累積和の作成ロジックの誤り}: 累積和作成時のループ範囲やインデックス参照にズレがある.
    \item \textbf{累積和を使用していない}: 累積和を使わず,クエリごとに配列をスライスして合計を計算する(TLE).
    \item \textbf{最終的な結果の計算ミス}: 累積和の値は取得できているが,その後の差分計算等のロジックが誤っている.
\end{enumerate}

\newpage

しかし, 図\ref{fig:010_pca_result}の左図(正解分布)を確認すると, 問題AGとは対照的に, 異なる誤答カテゴリ(特にカテゴリ2, 3, 4など)に対応する点が空間上で分離せず, 同一の領域(主に左下)に混在していることがわかる.
これは, これらの誤答がいずれも「配列確保とループ処理」という類似した構造を持っており, 静的解析では論理的な微差を特徴量として識別できなかったためである.
その結果, 右図(手法1)に示すように, 手法1は大半のデータを単一のクラスタ(C0)に統合してしまい, 有効な分類を行うことができなかった.

\begin{figure}[tbp]
    \centering
    \includegraphics[width=1.0\textwidth]{src/j_tea.png}
    \caption{問題Jにおける特徴量空間のPCA可視化結果(左:正解パターンの混在, 右:手法1による分類結果)}
    \label{fig:010_pca_result}
\end{figure}



\subsubsection{具体的な誤答コードの比較}
なぜ分類が困難であったかを確認するため,混同が生じた主な2つのグループについてコード構造の観点から比較を行う.

\paragraph{グループ1:アルゴリズム構造の類似(カテゴリ2, 3, 5)}\mbox{}\\
一つ目の混同パターンは,カテゴリ2(インデックス操作ミス),カテゴリ3(累積和作成ロジック誤り),カテゴリ5(計算ミス)が同一クラスタに含まれるケースである.
これらは「配列の確保」「前処理のループ」「クエリ処理のループ」という共通の構成を持っているため,静的解析特徴量が極めて類似する.

以下の3つのソースコード(\ref{list:010_index_error}, \ref{list:010_cumsum_error}, \ref{list:010_calc_error})は,それぞれ「累積和作成時」「クエリ処理時」「最終計算時」とバグの発生箇所が異なる.
しかし,静的解析ではファイル全体の特徴量として集約されるため,「どこで間違えているか」という局所的な情報の差異は無視され,同一視されてしまう.

\newpage

\begin{lstlisting}[caption={カテゴリ2:クエリ処理時のインデックス操作ミス}, label={list:010_index_error}, language=Python]
#...
Q = int(input())
for _ in range(Q):
    R, L = map(int, input().split())
    total1 = ps1[R] - ps1[L-1]
    # ...
    print(f"{total1} {total2}")
\end{lstlisting}

\begin{lstlisting}[caption={カテゴリ3:累積和の作成ロジックの誤り}, label={list:010_cumsum_error}, language=Python]
n = int(input())
p1 = [0] * (n+1)
# ...
for i in range(n):
     c, p = map(int, input().split())
     p1[i] = p1[i-1]
     if c == 1:
         p1[i] += p
     # ...
\end{lstlisting}

\begin{lstlisting}[caption={カテゴリ5:最終的な結果の計算ミス}, label={list:010_calc_error}, language=Python]
# ...
class1 = [0]
class2 = [0]
# ... 
q = int(input())
for _ in range(q):
    l, r = map(int, input().split())
    print(class1[r] - class1[l -1], class1[r] - class2[l - 1])
\end{lstlisting}

\paragraph{グループ2:単純ループ構造の類似(カテゴリ1, 4)}\mbox{}\\
二つ目の混同パターンは,カテゴリ1(インデックス以外のクエリ処理ミス)とカテゴリ4(累積和を使用していない)である.
ソースコード\ref{list:010_incomplete}とソースコード\ref{list:010_tle}に示すように,両者は共に「入力を受け取り,単純なループを回す」という構造を持つ.
複雑な分岐や計算を持たないため,サイクロマティック複雑度などの特徴量が共に低くなり,アルゴリズムとしての正誤(未完成か,計算量不足か)にかかわらず「単純なコード」としてまとめられてしまう傾向にある.

\newpage

\begin{lstlisting}[caption={カテゴリ1:インデックス以外のクエリ処理ミス}, label={list:010_incomplete}, language=Python]
# ...
q = int(input())
l = [0] * q
r = [0] * q
for i in range(q):
     l[i], r[i] = map(int, input().split())
\end{lstlisting}

\begin{lstlisting}[caption={カテゴリ4:累積和を使用していない(TLE)}, label={list:010_tle}, language=Python]
# ...
for i, j in L_list:
    A_sum = sum(A[i:j+1])
    B_sum = sum(B[i:j+1])
    # ...
\end{lstlisting}

%\subsubsection{考察}
以上の分析から,手法1の限界について以下の結論が得られる.

\begin{enumerate}
    \item 本手法はコードの「構造」を特徴量とするため,ロジックが異なっていても構文構造が似ていれば,同一のクラスタとみなされてしまう.
    \item カテゴリ2, 3, 5のように,「インデックスが1ずれている」「変数が逆である」といった行レベルの局所的な意味論的誤りは,大域的な静的特徴量には反映されにくい.
    \item カテゴリ1, 4のように,コード構造が単純な場合,特徴量空間での距離が縮まりやすく,実装意図の違いを識別できない.
\end{enumerate}

したがって,誤答の原因特定レベルの分類を実現するためには,誤りのパターンを区別可能な特徴量設計または,コードの意味を捉えるLLM等の手法との組み合わせが必要不可欠であると言える.


\subsection{手法2による分類が有効であった事例と考察}
\label{subsec:method2_success}

本節では,手法2の有効性について考察する.
手法2は,前節で述べた手法1では分類が困難であった事例に対しても,高い精度で誤答パターンを識別することに成功した.
そこで,まず「問題J」における手法1との比較を行い,続いて特に高い分類精度を示した「問題BI」および「問題X」の事例を通して,手法2が有効に機能した要因を分析する.

\subsubsection{問題Jにおける結果と考察}

手法1において,構文構造の類似性から誤分類が発生していた「問題J」に対し,手法2を適用したところ,分類精度の明確な向上が確認された.手法1では,インデックス操作ミス(カテゴリ2)と作成ロジック誤り(カテゴリ3)が,共に「ループによる配列操作」という構造的類似性を持つため同一クラスタに混同されていた.これに対し手法2は,コードの表層的な構造ではなく,誤りの意味内容を正しく解釈できたことで,両者を明確に分離することに成功した.重要な点は,LLMが単にバグの箇所を特定しただけでなく,「誤りの内容を正しく説明できた」ことが,結果として正しいカテゴリへの分類を導いたという事実である.

出力結果におけるLLMの判断プロセスを以下に示す.

\begin{itemize}
    \item \textbf{カテゴリ2(インデックス操作ミス)への分類}: 
    LLMは「累積和の作成自体は正しいが,最終的な区間計算の式において $L$ と $R$ が逆転する」という論理的な矛盾を言語化した. この「論理矛盾」という文脈を抽出できたため,構造が似ている他のカテゴリと区別し,正しくカテゴリ2へ分類できた.
    
    \item \textbf{カテゴリ3(作成ロジック誤り)への分類}: 
    p1[i] = p1[i-1] といった記述に対し,LLMは「配列外参照のリスク」や「累積和の漸化式としての不整合」を説明として生成した. この説明に基づき,計算式の間違いではなく前処理の段階での誤りであると判断され,正確にカテゴリ3へ分類された.
\end{itemize}

なお,本節では特徴的な事例として問題Jの特定カテゴリを取り上げたが,他の多くのカテゴリにおいても,同様にLLMが誤りの内容を正しく言語化し,適切な分類へと導いていることが確認された.
以上の結果から,誤りの特定能力は,それ自体が独立した成果というよりも,高精度な誤答分類を実現するための「意味的特徴量の抽出機能」として極めて有効に機能したと言える.手法1が捉えきれなかった「コードの実装意図と実際の挙動の乖離」をLLMが言語化できたことが,分類精度の向上に直接的に寄与したと結論付けられる.



\subsubsection{問題BI概要と結果の考察}
問題BI(061 - Deck \cite{typical90_bi})は,山札に対して条件に応じて「上に追加」「下に追加」「$x$番目を出力」という3種類のいずれかの操作を$Q$回行うシミュレーション問題である.
本問題の誤答コードは,表\ref{tab:error_categories}に示すように,主に以下の2つのカテゴリに分類された.

\begin{enumerate}
    \item \textbf{出力時のインデックス操作ミス}: クエリの指示通りに山札のインデックスを参照せず,入力された値そのものを出力してしまう等の誤り.
    \item \textbf{山札の操作・管理ミス}: 山札への追加操作(insertとappend)の挙動を逆にしてしまう等のロジック誤り.
\end{enumerate}

手法2で分類された具体的なコード例を以下に示す.
ソースコード\ref{list:bi_output_error}(カテゴリ1)とソースコード\ref{list:bi_operation_error}(カテゴリ2)は,どちらも制御構造は類似するが,誤りの性質が異なる.

ソースコード\ref{list:bi_output_error}では,配列 \texttt{X} を山札として操作しようとするが,出力処理において致命的なインデックス指定ミスがある.
本来は入力で与えられた「$x$ 番目」の要素を参照すべきところを,現在のクエリ番号であるループ変数 \texttt{i} をそのままインデックスとして使用しており,意図しない値を参照する.

一方,ソースコード\ref{list:bi_operation_error}においては,操作1と操作2の実装(appendとinsert)が逆であるというロジックの誤りがある.

\begin{lstlisting}[caption={カテゴリ1:出力時のインデックス操作ミス}, label={list:bi_output_error}, language=Python]
Q = int(input())
T = []
X = []

for i in range(Q):
  t, x = list(map(int,input().split()))
  T.append(t)
  X.append(x)

for i in range(Q):
  if T[i] == 1:
    X.insert(0, X[i])

  if T[i] == 2:
    X.insert(Q-1, X[i])

  if T[i] == 3:
    print(X[i])
\end{lstlisting}


\begin{lstlisting}[caption={カテゴリ2:山札の操作・管理ミス}, label={list:bi_operation_error}, language=Python]
Q = int(input())
s = []
ans = []

for i in range(Q):
    t, x = map(int, input().split())
    if t == 1:
        s.append(x)
    if t == 2:
        s.insert(0, x)
    if t == 3:
        ans.append(s[x-1])

for a in ans:
    print(a)
\end{lstlisting}

この事例における手法1と手法2の対比は示唆に富んでいる.
手法1では, 両者のコードは共に「条件分岐」と「リスト操作」の組み合わせで構成されており, 制御フローグラフや使用される演算子の種類といった構造的特徴が酷似するため, 同一クラスタに分類されてしまった.
一方で手法2は, 「山札の上」という自然言語の指示が「インデックス0への操作」に対応し, 「山札の下」が「末尾への操作」に対応するという意味的な対応関係を正しく認識できたため, 実装の矛盾を指摘できたと考えられる.


\subsubsection{問題X概要と結果の考察}
問題Xは, 長さ$N$の整数列$A$と$B$が与えられ, $A$の要素に対し「$+1$ または $-1$」する操作をちょうど $K$ 回行うことで, $A$を$B$に一致させることができるかを判定する問題である.
正解するためには, 以下の2つの条件を満たす必要がある.
\begin{enumerate}
    \item 差分の総和($\sum |A_i - B_i|$)が $K$ 以下であること.
    \item 残りの操作回数($K - \sum |A_i - B_i|$)が偶数であること.
\end{enumerate}

本問題の誤答は, 表\ref{tab:error_categories}に示すように, 主に「偶奇判定の記述漏れ」や「絶対値計算の欠落」などに分類された.

\paragraph{意味的解釈による分類の成功} \mbox{}\\
本問題においても, 手法2は高い分類精度を記録した. 特筆すべきは,「条件式の不足」という不可視の誤りを検出できた点である.

例えば, 「差分の総和が $K$ 以下か」のみを判定し, 偶奇判定を行っていない誤答コードにおいて,静的解析の観点からは, 正解コードと誤答コードは共に「ループで差分を計算し, 最後にif文で判定する」という同一の構造を持つため, この「条件式が一つ足りない」という差異を距離として捉えることは困難である.

しかし手法2は, 問題文中の「ちょうど $K$ 回」という記述から「過剰な回数分は $+1$ と $-1$ で相殺する必要があるため, 残回数は偶数でなければならない」という論理を導出し, コード内にその判定ロジックが存在しないことを指摘することで, 正しく「判定条件の記述漏れ」カテゴリへと分類した.

\subsubsection{有効性}
\label{subsubsec:discussion_explicit}

問題J,BI,Xなど, 手法2が特に有効であった事例を総括すると, ある共通点が浮かび上がる.
これらの問題では, 「問題文の指示」と「コードの実装」の対応関係が明確な点である.
問題BIでは,「上に追加」「下に追加」といった操作手順が, 具体的なデータ構造の操作と直結する.問題Xでは,「操作回数がちょうど $K$ 回」という条件が, 数学的な制約を厳密に規定する.
そのため, LLMはプロンプトとして与えられた問題文から期待されるロジックを構築し, それと提出コードとの間に生じている乖離を検出しやすいことが考えられる.



\subsection{手法2による分類が困難であった事例と考察}
\label{subsec:case_study_bo}

手法2は多くの問題で高い精度を示したが, 一部の問題では分類が安定しない, あるいは誤った推論に基づく分類が行われる事例が見られた.
本節では, その代表例として以下の2つの問題を取り上げ, LLMを用いた分類における課題を詳細に分析する.

\begin{itemize}
    \item \textbf{問題BO(067 - Base 8 to 9 \cite{typical90_bo})}:
    複数の誤りが混在するコードにおいて, LLMが「誤りの優先度」を正解ラベルの意図通りに判断できず, 分類結果が揺らいだ事例である.
    \item \textbf{問題V(022 - Cubic Cake \cite{typical90_v})}:
    LLMが数学的な事実やアルゴリズムの挙動に対して事実誤認を起こし, 誤った根拠に基づいて不適切なカテゴリへ分類してしまった事例である.
\end{itemize}

以下, 各問題について具体的なコード例を挙げながら, 分類失敗の要因とその背景にあるLLMの特性について考察する.
\subsubsection{問題BO概要と結果の考察}
問題BO(067 - Base 8 to 9 \cite{typical90_bo})は, 8進法の整数 $N$ に対し, 「10進数変換」「9進数変換」「数字8を5へ置換」「8進数とみなす」という一連の操作を $K$ 回行うシミュレーション問題である.
本問題の誤答コードは, 表\ref{tab:error_categories}に示すように, 主に以下のカテゴリに関連する誤りを含んでいる.

\begin{enumerate}
    \item \textbf{$N=0$ 等のコーナーケース処理漏れ}: 入力が0の場合のループ条件や出力の不備.
    \item \textbf{$n$進数変換処理の実装ミス}: 基数変換のロジック自体や, 桁数処理の固定化などの誤り.
\end{enumerate}

手法2による分類が困難であった具体的なコード例を以下に示す.
ソースコード\ref{list:bo_code_ambiguous}は, 一見すると単純な実装に見えるが, 分類器を迷わせる複数の要因を含んでいる.単なる条件分岐の漏れよりも, 「桁数を19桁に固定する」という実装方針の誤りの方がアルゴリズムとして致命的であると判断したので、正解ラベルではカテゴリ4($n$進数変換処理の実装ミス)に分類する.しかし, 手法2による判定では, $N=0$ の場合に停止しないという具体的な挙動に注目が集まりやすく, カテゴリ1(コーナーケース処理漏れ) へ誤分類される事例が多く見られた.我々の期待としては, 局所的なバグ($N=0$)よりも, コード全体を支配する根本的なロジックの誤りを優先度高く評価し, カテゴリ4へと導くことを理想としていたが, LLMにとっては両者の「誤りの重み」を比較判断することが困難であったと考えられる.

\newpage

\begin{lstlisting}[caption={カテゴリ1と4の境界:$N=0$ の処理漏れと固定桁ループ}, label={list:bo_code_ambiguous}, language=Python]
n, k = map(int,input().split())

for j in range(k):
    N = 0
    for l in range(19):
        N += n%10*(8**l)
        n = n//10
        if n == 0:
            break

    s = ""
    for i in range(19):
        s = str(N%9)+s
        N = N//9
    s = s.replace("8","5")
    n = int(s)

print(n)
\end{lstlisting}

\subsubsection{問題V概要と結果の考察}
問題V(022 - Cubic Cake \cite{typical90_v})は, 幅 $A$, 高さ $B$, 奥行き $C$ の直方体を, 各面に平行な切断のみですべて一辺が等しい立方体に分割する際の, 最小切断回数を求める問題である.
gをA,B,Cの最大公約数としたとき, 最小切断回数は以下の式で求められる.
\[
(A//g - 1) + (B//g - 1) + (C//g - 1)
\]

本問題の誤答コードは, 表\ref{tab:error_categories}に示すように, 主に以下のカテゴリに関連する誤りを含んでいる.
\begin{enumerate}
    \item \textbf{最大公約数の算出ミス}: ライブラリの誤用や自作関数のロジックミス.
    \item \textbf{浮動小数点演算による誤差}: 割り算(/)を使用してしまい, 桁落ちや誤差が生じる.
    \item \textbf{GCDを使用した場合の最小回数算出ミス}: 式の立て方や演算子の優先順位の誤り.
\end{enumerate}

手法2による分類において, 致命的な課題が確認された具体的なコード例を以下に示す.
ここでは, 「本当の誤り」を指摘できず, 「事実と異なる仕様」を根拠に誤分類を行った事例を取り上げる.

ソースコード\ref{list:v_hallucination_gcd}は, Python 3.9以降でサポートされている「3引数の \texttt{math.gcd}」を使用するが, 計算式の括弧不足という致命的なバグを含んでいる.
これに対しLLMは, 計算式の誤り(カテゴリ3)を指摘せず, 「\texttt{math.gcd} は引数を2つしか取れない」という事実誤認を行い, その結果として カテゴリ1(最大公約数の算出ミス) に誤分類した.

また, ソースコード\ref{list:v_hallucination_float}は, 浮動小数点除算を行っているため カテゴリ2(浮動小数点演算による誤差) に分類されるべき事例である.
しかしLLMは, 浮動小数点の危険性を看過して「正解コードである」と誤判定したり, あるいは全く関係のない「入出力のミス」等の理由を捏造して不適切なカテゴリへ分類する挙動を示した.

\begin{lstlisting}[caption={ハルシネーション:括弧不足を見逃し, 正しい構文を誤りと指摘}, label={list:v_hallucination_gcd}, language=Python]
import math

A,B,C = map(int,input().split())
n = math.gcd(A,B,C)

print((A+B+C//n)-3)
\end{lstlisting}

\begin{lstlisting}[caption={評価の揺らぎ:浮動小数点誤差の誤判定}, label={list:v_hallucination_float}, language=Python]
import math

A,B,C = map(int,input().split())
g = math.gcd(math.gcd(A, B), C)

print(int((A/g-1)+(B/g-1)+(C/g-1)))
\end{lstlisting}


\subsubsection{課題}
\label{subsec:method2_discussion_summary}

本節で詳述した2つの事例は, 手法2を用いたコード評価において, 単なる分類精度の数値からは読み取れない本質的な課題を浮き彫りにする.
まず, 問題BOの事例からは, 複合的な誤りに対する「優先度判断」の難しさが確認された.
実際の誤答コードでは, 「アルゴリズムの破綻」と「特定条件のバグ」が混在することが多い.
人間であれば前者をより重大な誤りとみなすが, LLMはその重み付けを適切に行えず, 表面的な挙動に引きずられて分類が揺らぐ傾向にある.
これは, 複雑な誤りを含むコードに対して, 単一の正解ラベルを求めようとする現在の評価定義自体の限界を示唆する.

さらに深刻な課題として, 問題Vでは「事実誤認」と「論理誤りの見落とし」という二重のリスクが顕在化した.
LLMは必ずしもコードの論理的な正当性を厳密に検証するわけではなく, 「括弧不足」という真の誤りを見逃しつつ, 「言語仕様に関する虚偽」を根拠に正しい記述を修正させようとする危険な挙動が確認された.
このような「もっともらしい嘘」を含んだフィードバックは, 学習者の知識習得を阻害するため, 教育システムへの導入において重大な障壁となる.

\section{本章まとめ}
\label{sec:chapter_summary_comparison}

本章で検証した手法1と手法2の有効性と課題を以下に整理する.

\begin{itemize}
    \item \textbf{手法1}
    \begin{itemize}
        \item \textbf{有効性}
        \begin{itemize}
            \item \textbf{構造の識別}: 全探索と数学的解法など, アルゴリズムの方針自体が異なるものを明確に分離できる.
            \item \textbf{再現性}: 常に一定の結果が得られる.
        \end{itemize}
        \item \textbf{課題}
        \begin{itemize}
            \item \textbf{微細な論理バグ}: ループ構造が同じであれば, インデックスのズレ等の局所的な誤りを識別できない.
        \end{itemize}
    \end{itemize}

    \item \textbf{手法2}
    \begin{itemize}
        \item \textbf{有効性}
        \begin{itemize}
            \item \textbf{意味の理解}: 変数名や文脈から, 実装者の意図や不可視の条件漏れを指摘できる.
        \end{itemize}
        \item \textbf{課題}
        \begin{itemize}
            \item \textbf{信頼性の欠如}: 複数のバグに対する優先度判断が揺らぐほか, 嘘の仕様を教えるハルシネーションのリスクがある.
        \end{itemize}
    \end{itemize}
\end{itemize}

\subsection{結論}

以上の検証から, 両手法はそれぞれ異なる役割と限界を持つことが明らかとなった.
手法1は, 問題Jのように微細な論理差分の検出には不向きであるものの, 問題AGのように学習者が選択した「アルゴリズムの方針」を構造的に分類することにおいては高い信頼性を持つ.
一方, 手法2は, 問題BIやXのように構造に現れない文脈的な誤りを検出可能であるが, その判定は確率的であり, 解釈の揺らぎや事実誤認のリスクを内包する.
したがって, 高精度な自動評価を実現するためには, 単に両者を並列に扱うのではなく, 手法1による「解法の大別(構造の固定)」を前提とし, その上で手法2による「論理の検証」を行うといった, 評価の粒度に応じた階層的なアプローチが必要であると結論付けられる.


\chapter{まとめと今後の課題}
\label{chap:conclusion}

\section{まとめ}

本研究では, プログラミング演習における指導負担の軽減とフィードバックの質的向上を目的として, 競プロ典型問題 90問星2の問題を対象とした自動分類手法を検討した.具体的に, 静的解析とクラスタリングを組み合わせた手法とLLMを用いた手法との比較検証を行った.

\section{今後の課題}
今後の課題として, 実用化に向けた精度と実行時間のバランス評価を行うことである.
また, 具体的なカテゴリ内容を与えず, カテゴリ数のみを指定してLLMに分類基準を生成させる手法を検討することである.
加えて, 本研究の「誤答分類」と卒業研究の「ヒント提案」を統合し, 教育支援システムとして実装することである.


\chapter*{謝辞}
本研究を行うにあたり,熱心にご指導頂いた山本晋一郎教授,大久保弘崇講師,粕谷英人講師に深く感謝致します.さらに,本研究に多大なご協力を頂いた山本研究室,大久保研究室,粕谷研究室の皆様に深く感謝致します.

\begin{thebibliography}{9}
  \bibitem{Asanas Cluster}Paiva, J.C., Leal, J.P., and Figueira, Á., Clustering source code from automated assessment of programming assignments. International Journal of Data Science and Analytics,  Volume 20, pages 1581–1592, (2025)
  
  \bibitem{fujiwara2022}藤原 裕士 “深層学習を用いたソースコード分類に関する研究”. \url{https://sel.ist.osaka-u.ac.jp/lab-db/Dthesis/archive/40/40.pdf}, (2022)%{\today}"閲覧"

  \bibitem{openai_structured_2024} OpenAI, ``Introducing Structured Outputs in the API,'' 2024. \url{https://openai.com/index/introducing-structured-outputs-in-the-api/} \today 閲覧.

  \bibitem{vellum_2025} Anita Kirkovska, ``Claude 3.5 Sonnet vs GPT-4o: Comprehensive Comparison,'' Vellum.ai, 2025. \url{https://www.vellum.ai/blog/claude-3-5-sonnet-vs-gpt4o} \today 閲覧.

  \bibitem{sentisight_2025} Alius Noreika, ``Claude 3.5 Sonnet vs. GPT-4o: The Ultimate Comparison,'' SentiSight.ai, 2025. \url{https://www.sentisight.ai/claude-3-5-sonnet-vs-gpt-4o-ultimate-comparison/} \today 閲覧.

% コンテストトップページ
  \bibitem{typical90_top} AtCoder. ``競プロ典型 90 問''. \url{https://atcoder.jp/contests/typical90/} (\today 閲覧).

  % 問004 -> d
  \bibitem{typical90_d} AtCoder. ``004 - Cross Sum''. \url{https://atcoder.jp/contests/typical90/tasks/typical90_d} (\today 閲覧).

  % 問010 -> j
  \bibitem{typical90_j} AtCoder. ``010 - Score Sum Queries''. \url{https://atcoder.jp/contests/typical90/tasks/typical90_j} (\today 閲覧).

  % 問022 -> v
  \bibitem{typical90_v} AtCoder. ``022 - Cubic Cake''. \url{https://atcoder.jp/contests/typical90/tasks/typical90_v} (\today 閲覧).

  % 問024 -> x
  \bibitem{typical90_x} AtCoder. ``024 - Select +/- One''. \url{https://atcoder.jp/contests/typical90/tasks/typical90_x} (\today 閲覧).

  % 問027 -> aa
  \bibitem{typical90_aa} AtCoder. ``027 - Sign Up Requests''. \url{https://atcoder.jp/contests/typical90/tasks/typical90_aa} (\today 閲覧).

  % 問033 -> ag
  \bibitem{typical90_ag} AtCoder. ``033 - Not Too Bright''. \url{https://atcoder.jp/contests/typical90/tasks/typical90_ag} (\today 閲覧).

  % 問055 -> bc
  \bibitem{typical90_bc} AtCoder. ``055 - Select 5''. \url{https://atcoder.jp/contests/typical90/tasks/typical90_bc} (\today 閲覧).
  
  % 問061 -> bi
  \bibitem{typical90_bi} AtCoder. ``061 - Deck''. \url{https://atcoder.jp/contests/typical90/tasks/typical90_bi} (\today 閲覧).

  % 問067 -> bo
  \bibitem{typical90_bo} AtCoder. ``067 - Base 8 to 9''. \url{https://atcoder.jp/contests/typical90/tasks/typical90_bo} (\today 閲覧).

  % 問078 -> bz
  \bibitem{typical90_bz} AtCoder. ``078 - Easy Graph Problem''. \url{https://atcoder.jp/contests/typical90/tasks/typical90_bz} (\today 閲覧).

\end{thebibliography}

% \appendix
% 付録

% \chapter{xxxのソースコード}
% \lstinputlisting[caption=xxx.ts, label=src]{ファイルパス}

% \chapter{xxxソースコード}
% \lstinputlisting[caption=xxx.ts]{ファイルパス}

\end{document}
