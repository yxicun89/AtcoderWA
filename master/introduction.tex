\chapter{はじめに}

\section{背景}
プログラミング技術の習得は,現代社会においてますます重要性を増す.
その教育手法として,プログラミング演習やコンテストが活用されている.
これらの演習において,学習者が提出した大量の誤答コードを効率的に分析し,
適切なフィードバックを提供することは,学習効果の向上および指導者の負担軽減において重要な課題である.
しかし現状では,この誤答分析の多くは指導者の経験則に基づいた手作業で行われている.
卒業研究において,プログラミング演習における誤答コードの手動分類を試みた際にも,
その作業が非常に膨大な時間を要するだけでなく,分類基準が作業者の主観に依存,客観性を担保することが困難である課題があった.
したがって,プログラミング指導の効率化および質の均一化のためには,誤答分類作業を自動化する仕組みの構築が強く求められる.

\section{目的}
本研究の目的は,1.1で述べた誤答分類の自動化という課題を解決し,
それによりプログラミング学習の効果向上と指導者の負担軽減に寄与することにある.
本研究は,この課題の解決にあたり,学習データを必要とせず,かつコードの論理構造を決定論的に解析できる静的解析というアプローチに着目する.
上記のアプローチに基づき,本研究ではコードの静的解析と非階層クラスタリングを組み合わせた手法を提案し,その有効性を検証する.
さらに,この処理フローを解析から機械学習までを同一環境で行うプログラミング環境で構築することを目指す.
これにより,工学的な導入の容易さや,将来的に多様な機械学習手法へ展開する際の拡張性といった利点が得られると期待される.

\section{本論文の構成}
第2章で先行研究・関連研究を述べる.
%
第3章で提案する誤答分類の手法を述べる.
%
第4章で提案手法の効果を示すための実験方法を述べる.
%
第5章で実験結果について評価する.
%
第6章でまとめと今後の課題について述べる.


